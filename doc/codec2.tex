\documentclass{article}
\usepackage{amsmath}
\usepackage{hyperref}
\usepackage{tikz}

\usepackage{xstring}
\usepackage{catchfile}

\CatchFileDef{\headfull}{../.git/HEAD}{}
\StrGobbleRight{\headfull}{1}[\head]
\StrBehind[2]{\head}{/}[\branch]
\IfFileExists{../.git/refs/heads/\branch}{%
    \CatchFileDef{\commit}{../.git/refs/heads/\branch}{}}{%
    \newcommand{\commit}{\dots~(in \emph{packed-refs})}}
\newcommand{\gitrevision}{%
  \StrLeft{\commit}{7}%
}

\title{Codec 2}
\author{David Rowe\\ \\ Revision: {\gitrevision} on branch: {\branch}}

\begin{document}
\maketitle

\section{Introduction}

Codec 2 is an open source speech codec designed for communications quality speech between 700 and 3200 bit/s. The main application is low bandwidth HF/VHF digital radio. It fills a gap in open source voice codecs beneath 5000 bit/s and is released under the GNU Lesser General Public License (LGPL).  It is written in C99 standard C.

The Codec 2 project was started in 2009 in response to the problem of closed source, patented, proprietary voice codecs in the sub-5 kbit/s range, in particular for use in the Amateur Radio service.

This document describes Codec 2 at two levels.  Section \ref{sect:overview} is a high level overview aimed at the Radio Amateur, while Section \ref{sect:details} contains a more detailed description with math and signal processing theory. This document is not a concise algorithmic description, instead the algorithm is defined by the reference C99 source code and automated tests (ctests).

This production of this document was kindly supported by an ARDC grant \cite{ardc2023}.  As an open source project, many people have contributed to Codec 2 over the years - we deeply appreciate all of your support.

\section{Codec 2 for the Radio Amateur}
\label{sect:overview}

\section{Signal Processing Details}
\label{sect:details}

\section{Further Work}


\cite{griffin1988multiband}

\bibliographystyle{plain}
\bibliography{codec2_refs}
\end{document}
